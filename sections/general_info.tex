\chapter{General Information}

\section{Reading in data files}
When you are trying to cluster and classify data, typically the first step is reading in the data from a file. There are several different ways that data is typically stored in a data file, the most common formats are comma separated values (csv) or tab separated values (tsv). For both of these types of files, R has some inbuilt functions for reading them into data frames.

Typically csv files tend to be easier to read in, as they usually have proper column labels and very simple value separation. Tsv files can vary in their formatting between tabs and different types of spacing, and often do not have column labels.

\subsection{CSV files}
Csv files are very easy to read in with R. Let's consider the following csv file.

\begin{verbatim}
id,mag,amp,error
00001,20.123,19.312,0.2486362
00003,19.02,20.2,0.47823
00004,18.8,21.389,0.277669
\end{verbatim}

In R you can read in csv files using the \verb|read.csv| function. We must note that this csv file has column headers, so we need to add an argument to indicate this.

\begin{verbatim}
myData <- read.csv("data.csv", header=TRUE)
\end{verbatim}

\subsection{TSV files}
Let's take a look at a more unusual tsv file and read it in to R. This tsv file uses a special format that allocates a specific width to each column, such that a non-standard amount of additional spaces are added for padding.

\begin{verbatim}
00001  20.123 19.312  0.2486362
00003  19.023 20.213  0.4782373
00004  18.789 21.389  0.2776688
\end{verbatim}

To read this data in with R we need to use the \verb|read.table| function. We must note that the data file does not have column headers, and does not have a sepcific separater string.

\begin{verbatim}
myData <- read.table("data.tsv")
\end{verbatim}

The resulting data frame will include all of the given data, however its column names with be generic. The next thing we will need to do is to label the columns. In R there are many different ways to rename columns, however since we want to rename all of the columns we can make use of the \verb|names| function.

\begin{verbatim}
names(myData) <- c("id", "mag", "amp", "error")
\end{verbatim}

\section{Normalizing data}
Before you do any sort of clustering on data, you should first normalize the data. Datasets generally have features that are of different units and ranges of values and this must be taken into account when performing clustering.

To normalize a data frame in R, you can use the \verb|scale| function.

\begin{verbatim}
normalizedData <- scale(myData)
\end{verbatim}

\section{Misc. Functions}

\subsection{Conversion functions}
To convert a numerical factor into the corresponding num representations you can use a combination of the \verb|as.character| and \verb|as.numeric| functions. Any values that are not able to be converted into numeric values will be converted into \verb|NA|s.

\begin{verbatim}
myData$phi31 <- as.numeric(as.character(myData$phi31))
\end{verbatim}

\subsection{Mapping functions}
To map a function over the rows of a data frame and be able to access specific columns you can use the \verb|with| function.

\begin{verbatim}
myData$logMag <- with(myData, log(mag))
\end{verbatim}